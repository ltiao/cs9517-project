%%%%%%%%%%%%%%%%%%%%%%%%%%%%%%%%%%%%%%%%%
% Short Sectioned Assignment
% LaTeX Template
% Version 1.0 (5/5/12)
%
% This template has been downloaded from:
% http://www.LaTeXTemplates.com
%
% Original author:
% Frits Wenneker (http://www.howtotex.com)
%
% License:
% CC BY-NC-SA 3.0 (http://creativecommons.org/licenses/by-nc-sa/3.0/)
%
%%%%%%%%%%%%%%%%%%%%%%%%%%%%%%%%%%%%%%%%%

%----------------------------------------------------------------------------------------
% PACKAGES AND OTHER DOCUMENT CONFIGURATIONS
%----------------------------------------------------------------------------------------

\documentclass[11pt]{article} % A4 paper and 11pt font size

\usepackage[a4paper,margin=0.8in,footskip=.3in]{geometry}

\linespread{1.0} % Line spacing

\setlength\parindent{0pt} % Removes all indentation from paragraphs - comment this line for an assignment with lots of text
\setlength{\parskip}{6pt}

\usepackage{sectsty} % Allows customizing section commands
\allsectionsfont{\centering \normalfont\scshape} % Make all sections centered, the default font and small caps

\usepackage{fancyhdr} % Custom headers and footers
\pagestyle{fancyplain} % Makes all pages in the document conform to the custom headers and footers
\fancyhead{} % No page header - if you want one, create it in the same way as the footers below
\fancyfoot[L]{} % Empty left footer
\fancyfoot[C]{} % Empty center footer
\fancyfoot[R]{\thepage} % Page numbering for right footer
\renewcommand{\headrulewidth}{0pt} % Remove header underlines
\renewcommand{\footrulewidth}{0pt} % Remove footer underlines
\setlength{\headheight}{13.6pt} % Customize the height of the header

\usepackage[utf8]{inputenc}
\usepackage[T1]{fontenc} % Use 8-bit encoding that has 256 glyphs

\usepackage{fourier} % Use the Adobe Utopia font for the document - comment this line to return to the LaTeX default
\usepackage[english]{babel} % English language/hyphenation

\usepackage{amsmath,amsfonts,amsthm} % Math packages
\theoremstyle{plain}
\newtheorem{theorem}{Theorem}[section]
\newtheorem{corollary}{Corollary}[theorem]
\newtheorem{proposition}[theorem]{Proposition}
\newtheorem{lemma}[theorem]{Lemma}

\theoremstyle{definition}
\newtheorem{definition}{Definition}[section]

\theoremstyle{remark}
\newtheorem*{remark}{Remark}

\numberwithin{equation}{section} % Number equations within sections (i.e. 1.1, 1.2, 2.1, 2.2 instead of 1, 2, 3, 4)
\numberwithin{figure}{section} % Number figures within sections (i.e. 1.1, 1.2, 2.1, 2.2 instead of 1, 2, 3, 4)
\numberwithin{table}{section} % Number tables within sections (i.e. 1.1, 1.2, 2.1, 2.2 instead of 1, 2, 3, 4)

\usepackage{algpseudocode}
\usepackage{algorithm}

\usepackage{listings}

\usepackage{tikz}

\usepackage{natbib}
\bibliographystyle{plainnat}

\usepackage{lipsum} % Used for inserting dummy 'Lorem ipsum' text into the template

\usepackage{siunitx}

\usepackage{float}
\usepackage{wrapfig}
\usepackage{graphicx}

\usepackage{multirow}
\usepackage{booktabs}
\usepackage{array}

\usepackage{url}

\usepackage{hyperref}
\hypersetup{
    colorlinks,
    citecolor=black,
    filecolor=black,
    linkcolor=black,
    urlcolor=black
}
\usepackage{cleveref}
\usepackage{microtype}

\usepackage{todonotes}

\usepackage{titling}
\setlength{\droptitle}{-5em}

%----------------------------------------------------------------------------------------
% TITLE SECTION
%----------------------------------------------------------------------------------------

\newcommand{\horrule}[1]{\rule{\linewidth}{#1}} % Create horizontal rule command with 1 argument of height

\title{ 
\normalfont \normalsize 
% \textsc{university, school or department name} \\ [25pt] % Your university, school and/or department name(s)
\horrule{0.5pt} \\[0.4cm] % Thin top horizontal rule
\Large Computer Vision Project (Part 2) \\ [0.1cm] % The assignment title
\large COMP9517 - Semester 1, 2015 \\ [0.2cm]
\horrule{2pt} \\[0.5cm] % Thick bottom horizontal rule
}

\author{
  Louis Tiao \\
  (\texttt{3390558})
  \and
  Edward Lee\\
  (\texttt{3376371})
} % Your name

\date{\normalsize\today} % Today's date or a custom date

\begin{document}

\maketitle % Print the title

\section{Overview}

Facial recognition is an example of a task which is relatively easy for most humans but much more difficult for computers. In this project, we explore and implement several conventional methods for facial recognition and assess their suitability for specific facial recognition applications.

\section{Project Scope}

\begin{enumerate}
  \item \textbf{Face detection / localization:} For facial recognition to be applied to general images, it is first necessary to solve the problem of finding the locations and sizes of any faces appearing in the image.
  \item \textbf{Facial recognition:} The main goal of this project is to survey and implement some of the state-of-art facial recognition techniques for solving the problem of automatically identifying or verifying human faces from a selection of detected faces in a digital image or frame from a video sequence. We will implement and benchmark some of the most wide-used methods in the literature outlined in \cref{sec:lit_survey} and provide comparisons and analyses on their relative performance.
\end{enumerate}

\section{Literature Survey} \label{sec:lit_survey}

An overview of the most widely-used techniques for both face detection and facial recognition can be found in Sections 14.1.1 and 14.2 of the prescribed computer vision textbook \citep[p.~658,~668]{szeliski2010computer} respectively. We present a survey for the subtasks of facial detection and recognition respectively.

\subsection{Face Detection Methods} 

An extensive literature survey of the wide variety of fast face detection methods is given in \citep{yang2002detecting}. According to their taxonomy,
the face detection techniques can be classified as\citep{szeliski2010computer}:

\begin{itemize}
  \item \textbf{feature-based}: ``attempts to find the locations of distinctive image features such as the eyes, nose, and mouth, and then verify whether these features are in a plausible geometrical arrangement.''
  \item \textbf{appearance-based}: ``scan over small overlapping rectangular patches of the image searching for likely face candidates, which can then be refined using a cascade of more expensive but selective detection algorithms''
  \item \textbf{template-based}: active appearance models (AAMs), which we will not investigate in our project.
\end{itemize}

We will mainly focus on the appearance-based approaches, namely

\begin{itemize}
  \item \textbf{Boosting}: ``Of all the face detectors currently in use, the one introduced by \citep{viola2004robust} is probably the best known and most widely used. Their technique was the first to introduce the concept of boosting to the computer vision community, which involves training a series of increasingly discriminating simple classifiers and then blending their outputs''
  \item \textbf{Support Vector Machines}: \citep{osuna1997training}
\end{itemize}

\subsection{Facial Recognition Methods}

A comprehensive treatment of facial recognition methods can be found in the \emph{Handbook of Face Recognition} \citep{jain2005handbook}.

``While the earlier approaches to face recognition involved finding the locations of distinctive image features, such as the eyes, nose, and mouth, and measuring the distances between these feature locations, the more recent approaches are largely dominated by those which rely on comparing gray-level images projected onto lower dimensional subspaces called eigenfaces.''

\begin{itemize}
  \item \textbf{Eigenfaces}: \citep{turk1991eigenfaces,turk1991face}
  \item \textbf{Fisherfaces}: ``In their paper, \citep{belhumeur1997eigenfaces} show that Fisherfaces significantly outperform the original eigenfaces algorithm, especially when faces have large amounts of illumination variation.''
  \item \textbf{Modular eigenfaces}: ``Another way to improve the performance of eigenface-based approaches is to break up the image into separate regions such as the eyes, nose, and mouth (Figure 14.18) and to match each of these modular eigenspaces independently \citep{pentland1994view}. The advantage of such a modular approach is that it can tolerate a wider range of viewpoints, because each part can move relative to the others. It also supports a larger variety of combinations, e.g., we can model one person as having a narrow nose and bushy eyebrows, without requiring the eigenfaces to span all possible combinations of nose, mouth, and eyebrows.''
\end{itemize}

\subsection{Datasets}

The relevant datasets for facial recognition are provided by 
\citep[p.~719]{szeliski2010computer} and reproduced in the Appendix below.

\section{Approach}

\begin{enumerate}
\item \textbf{Facial Detection:} With the packages such as OpenCV we concentrate on the following appearance-based techniques for facial detection:
  \begin{itemize}
    \item Boosting
    \item SVMs
  \end{itemize}

    \item \textbf{Facial Recognition:} From the datasets found in \cref{tab:datasets} we will implement, with the aid of OpenCV libraries the following linear algebra approaches to facial recognition:
    \begin{itemize}
      \item Eigenfaces
      \item Fisherfaces
      \item Modular Eigenfaces
    \end{itemize}
\end{enumerate}

\subsection{Evaluation}

\begin{enumerate}
  \item \textbf{Facial Detection:} In order to evaluate each facial detection algorithm, we will construct a series of images which contain faces of different orientations in different locations in the frame of each image. The metric we will use to evaluate the accuracy of a particular technique will be the percentage of correctly detected images.
    \item \textbf{Facial Recognition} Most datasets consist of multiple subjects, of which there exists multiple images. For every individual, we split up their images into a training set and an evaluation set. This is so that when we match an image from the evaluation set to images in the training set, the actual image should resemble a known subject, but the image will not have been previously seen. Then we can evaluate the accuracy of our facial recognition system by considering the percentage of correctly recognized images.
\end{enumerate}

% \begin{enumerate}
%   \item \textbf{Facial Recognition System:} We wish to implement more than one facial recognition system in order to do comparisons and evaluations on state of the art facial recognition techniques. One common method real-time method from involves the use of Eigenfaces, a low-dimensional representation of facial images, for facial recognition \citep{turk1991eigenfaces}. A more novel method described in \citep{krivzaj2010adaptation} details the use of SIFT feature detection in facial detection. In Part 1 of our project we already used SIFT feature detection on images meaning that we can adapt previous work and compare it to more advanced techniques such as eigenfaces for facial recognition. We will be using the extended yale face database B to begin our investigation, acquiring more datasets as needed. 

%   \item \textbf{Facial Verification/Facial Search Engine:} 
%   Upon completion of a facial recognition system, we can then adapt it to display 
% \end{enumerate}


\section{Plan}

% general plan here

\begin{figure}[H]
\centering
\begin{tikzpicture}[every node/.style={font=\normalsize,
  minimum height=0.5cm,minimum width=0.5cm},]

  \draw[step=2cm,gray,very thin,dashed] (0,0) grid (10,8);
  \node[] at (1,-0.5) {Week 9};
  \node[] at (3,-0.5) {Week 10};
  \node[] at (5,-0.5) {Week 11};
  \node[] at (7,-0.5) {Week 12};
  \node[] at (9,-0.5) {Week 13};

  % datasets
  \node[] at (-3, 7) {Gather datasets};
  \fill[green!40!white] (0,6.5) rectangle (2,7.5);

  % SIFT feature detection + Eigen faces  
  \node[] at (-3, 5) {Facial Detection};
  \fill[blue!40!white] (0,4.5) rectangle (4,5.5);
  \fill[red!40!white] (4,4.5) rectangle (6,5.5);

  % ranking
  \node[] at (-3, 3) {Facial Recognition};
  \fill[blue!40!white] (4,2.5) rectangle (6,3.5);
  \fill[red!40!white] (6,2.5) rectangle (8,3.5);

  % documentation
  \node[] at (-3, 1) {Report and Presentation};
  \fill[green!40!white] (6,0.5) rectangle (10,1.5);

  % legend
  \fill[green!40!white] (-4, -2) rectangle (0, -1);
  \node[] at (-2, -1.5) {Research / Writing};
  \fill[blue!40!white] (0, -2) rectangle (4, -1);
  \node[] at (2, -1.5) {Implementation};
  \fill[red!40!white] (4, -2) rectangle (8, -1);
  \node[] at (6, -1.5) {Benchmarking};

\end{tikzpicture}
\end{figure}
\bibliography{../bibliography}

\section*{Appendix}

\begin{table}[H]
  \centering
  \caption{Facial recognition datasets}
  \label{tab:datasets}
  \begin{tabular}{ l p{10cm} }
    Name & Description \\ 
    \hline
    Yale face database & Centered face images (Frontal faces) \\
    & \url{http://www1.cs.columbia.edu/~belhumeur/} \\
    FERET & Centered face images (Frontal faces) \\
    & \url{http://www.frvt.org/FERET} \\
    FRVT & Centered face images (Faces in various poses) \\
    & \url{http://www.frvt.org/} \\
    CMU PIE database & Centered face images (Faces in various poses) \\
    & \url{http://www.ri.cmu.edu/projects/project_418.html} \\
    CMU Multi-PIE database & Centered face images (Faces in various poses) \\
    & \url{http://multipie.org} \\
    Faces in the Wild & Internet images (Faces in various poses) \\
    & \url{http://vis-www.cs.umass.edu/lfw/} \\
    CMU frontal faces & Patches (Frontal faces) \\
    & \url{http://vasc.ri.cmu.edu/idb/html/face/frontal images} \\
    MIT frontal faces & Patches (Frontal faces) \\
    & \url{http://cbcl.mit.edu/software-datasets/FaceData2.html} \\
    CMU face detection databases & Multiple faces (Faces in various poses) \\
    & \url{http://www.ri.cmu.edu/research project detail.html?project id=419}
  \end{tabular}
\end{table}



%----------------------------------------------------------------------------------------

\end{document}