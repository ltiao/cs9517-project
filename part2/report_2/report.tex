%%%%%%%%%%%%%%%%%%%%%%%%%%%%%%%%%%%%%%%%%
% Short Sectioned Assignment
% LaTeX Template
% Version 1.0 (5/5/12)
%
% This template has been downloaded from:
% http://www.LaTeXTemplates.com
%
% Original author:
% Frits Wenneker (http://www.howtotex.com)
%
% License:
% CC BY-NC-SA 3.0 (http://creativecommons.org/licenses/by-nc-sa/3.0/)
%
%%%%%%%%%%%%%%%%%%%%%%%%%%%%%%%%%%%%%%%%%

%----------------------------------------------------------------------------------------
%	PACKAGES AND OTHER DOCUMENT CONFIGURATIONS
%----------------------------------------------------------------------------------------

\documentclass[11pt]{article} % A4 paper and 11pt font size

\usepackage[a4paper,margin=1in,footskip=.3in]{geometry}

\linespread{1.0} % Line spacing

\setlength\parindent{0pt} % Removes all indentation from paragraphs - comment this line for an assignment with lots of text
\setlength{\parskip}{6pt}

\usepackage{sectsty} % Allows customizing section commands
\allsectionsfont{\centering \normalfont\scshape} % Make all sections centered, the default font and small caps

\usepackage{fancyhdr} % Custom headers and footers
\pagestyle{fancyplain} % Makes all pages in the document conform to the custom headers and footers
\fancyhead{} % No page header - if you want one, create it in the same way as the footers below
\fancyfoot[L]{} % Empty left footer
\fancyfoot[C]{} % Empty center footer
\fancyfoot[R]{\thepage} % Page numbering for right footer
\renewcommand{\headrulewidth}{0pt} % Remove header underlines
\renewcommand{\footrulewidth}{0pt} % Remove footer underlines
\setlength{\headheight}{13.6pt} % Customize the height of the header

\usepackage[utf8]{inputenc}
\usepackage[T1]{fontenc} % Use 8-bit encoding that has 256 glyphs

\usepackage{fourier} % Use the Adobe Utopia font for the document - comment this line to return to the LaTeX default
\usepackage[english]{babel} % English language/hyphenation

\usepackage{amsmath,amsfonts,amsthm} % Math packages
\theoremstyle{plain}
\newtheorem{theorem}{Theorem}[section]
\newtheorem{corollary}{Corollary}[theorem]
\newtheorem{proposition}[theorem]{Proposition}
\newtheorem{lemma}[theorem]{Lemma}

\theoremstyle{definition}
\newtheorem{definition}{Definition}[section]

\theoremstyle{remark}
\newtheorem*{remark}{Remark}

\numberwithin{equation}{section} % Number equations within sections (i.e. 1.1, 1.2, 2.1, 2.2 instead of 1, 2, 3, 4)
\numberwithin{figure}{section} % Number figures within sections (i.e. 1.1, 1.2, 2.1, 2.2 instead of 1, 2, 3, 4)
\numberwithin{table}{section} % Number tables within sections (i.e. 1.1, 1.2, 2.1, 2.2 instead of 1, 2, 3, 4)

\usepackage{algpseudocode}
\usepackage{algorithm}

\usepackage{listings}

\usepackage{tikz}

\usepackage{natbib}
\bibliographystyle{plainnat}

\usepackage{lipsum} % Used for inserting dummy 'Lorem ipsum' text into the template

\usepackage{siunitx}

\usepackage{float}
\usepackage{wrapfig}
\usepackage{graphicx}

\usepackage{multirow}
\usepackage{booktabs}
\usepackage{array}

\usepackage{url}

\usepackage{hyperref}
\hypersetup{
    colorlinks,
    citecolor=black,
    filecolor=black,
    linkcolor=black,
    urlcolor=black
}
\usepackage{cleveref}
\usepackage{microtype}

\usepackage{todonotes}

% \usepackage{titling}
% \setlength{\droptitle}{-5em}

%----------------------------------------------------------------------------------------
%	TITLE SECTION
%----------------------------------------------------------------------------------------

\newcommand{\horrule}[1]{\rule{\linewidth}{#1}} % Create horizontal rule command with 1 argument of height

\title{	
\normalfont \normalsize 
% \textsc{university, school or department name} \\ [25pt] % Your university, school and/or department name(s)
\horrule{0.5pt} \\[0.4cm] % Thin top horizontal rule
\Large Computer Vision Project (Part 2) \\ [0.1cm] % The assignment title
\large COMP9517 - Semester 1, 2015 \\ [0.2cm]
\horrule{2pt} \\[0.5cm] % Thick bottom horizontal rule
}

\author{
	Louis Tiao \\
	(\texttt{3390558})
	\and
	Edward Lee\\
	(\texttt{3376371})
} % Your name

\date{\normalsize\today} % Today's date or a custom date

\begin{document}

\maketitle % Print the title

\section{Overview}

\section{Problem Statement}

\todo{Need to nail down a subset the below problems.}

\begin{itemize}
	\item Detecting the fact or state of one or more faces appearing in a 
		digital image
  \item Face verification - confidence score of similarity between faces (biometric security)
	\item Facial search engine - searching a database of images or videos against a query face.
  \item Head pose estimation
  \item Unsupervised learning - Clustering faces given a number of faces
  \item Supervised learning - classification problems depending on availability of labeled images
    \begin{itemize}
      \item Age, Gender
      \item Mood
      \item Attractiveness (could build a service that leverages Tinder API, where our software
      creates a database of attractive and unattractive faces based on right and left swipes respective
      by computing eigenfaces, and then automates swipes once enough examples have been seen)
    \end{itemize}
\end{itemize}

\section{Literature Survey}

Cited works \citep{litvin2003,battiato2007sift,grundmann2011auto,liu2009content,Matsushita2006}

Section 14.2 Szeliski

http://www.face-rec.org/

\section{Approach}

% outline subtasks/problems here

\begin{itemize}
  \item feature-based
  \item template-based
  \item appearance-based
\end{itemize}

Eigenfaces (Section 14.2.1 Szeliski)

% mention which datasets to use, etc.

Page 719 Szeliski

\section{Plan}

% general plan here

\begin{description}
  \item[Week 9] \hfill \\
  Do stuff
  \item[Week 10] \hfill \\
  Do stuff
  \item[Week 11] \hfill \\
  Do stuff
  \item[Week 12] \hfill \\
  Do stuff
  \item[Week 13] \hfill \\
  Do more stuff \ldots
\end{description}

\bibliography{../bibliography}

%----------------------------------------------------------------------------------------

\end{document}