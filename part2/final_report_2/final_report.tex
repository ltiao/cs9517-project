%%%%%%%%%%%%%%%%%%%%%%%%%%%%%%%%%%%%%%%%%
% Short Sectioned Assignment
% LaTeX Template
% Version 1.0 (5/5/12)
%
% This template has been downloaded from:
% http://www.LaTeXTemplates.com
%
% Original author:
% Frits Wenneker (http://www.howtotex.com)
%
% License:
% CC BY-NC-SA 3.0 (http://creativecommons.org/licenses/by-nc-sa/3.0/)
%
%%%%%%%%%%%%%%%%%%%%%%%%%%%%%%%%%%%%%%%%%

%----------------------------------------------------------------------------------------
% PACKAGES AND OTHER DOCUMENT CONFIGURATIONS
%----------------------------------------------------------------------------------------

\documentclass[11pt, twocolumn]{article} % A4 paper and 11pt font size

\usepackage[a4paper,margin=0.8in,footskip=.3in]{geometry}

\linespread{1.0} % Line spacing

\setlength\parindent{0pt} % Removes all indentation from paragraphs - comment this line for an assignment with lots of text
\setlength{\parskip}{6pt}

\usepackage{sectsty} % Allows customizing section commands
\allsectionsfont{\centering \normalfont\scshape} % Make all sections centered, the default font and small caps

\usepackage{fancyhdr} % Custom headers and footers
\pagestyle{fancyplain} % Makes all pages in the document conform to the custom headers and footers
\fancyhead{} % No page header - if you want one, create it in the same way as the footers below
\fancyfoot[L]{} % Empty left footer
\fancyfoot[C]{} % Empty center footer
\fancyfoot[R]{\thepage} % Page numbering for right footer
\renewcommand{\headrulewidth}{0pt} % Remove header underlines
\renewcommand{\footrulewidth}{0pt} % Remove footer underlines
\setlength{\headheight}{13.6pt} % Customize the height of the header

\usepackage[utf8]{inputenc}
\usepackage[T1]{fontenc} % Use 8-bit encoding that has 256 glyphs

\usepackage{fourier} % Use the Adobe Utopia font for the document - comment this line to return to the LaTeX default
\usepackage[english]{babel} % English language/hyphenation

\usepackage{amsmath,amsfonts,amsthm} % Math packages
\theoremstyle{plain}
\newtheorem{theorem}{Theorem}[section]
\newtheorem{corollary}{Corollary}[theorem]
\newtheorem{proposition}[theorem]{Proposition}
\newtheorem{lemma}[theorem]{Lemma}

\theoremstyle{definition}
\newtheorem{definition}{Definition}[section]

\theoremstyle{remark}
\newtheorem*{remark}{Remark}

\numberwithin{equation}{section} % Number equations within sections (i.e. 1.1, 1.2, 2.1, 2.2 instead of 1, 2, 3, 4)
\numberwithin{figure}{section} % Number figures within sections (i.e. 1.1, 1.2, 2.1, 2.2 instead of 1, 2, 3, 4)
\numberwithin{table}{section} % Number tables within sections (i.e. 1.1, 1.2, 2.1, 2.2 instead of 1, 2, 3, 4)

\usepackage{algpseudocode}
\usepackage{algorithm}

\usepackage{listings}

\usepackage{tikz}

\usepackage{natbib}
\bibliographystyle{plainnat}

\usepackage{lipsum} % Used for inserting dummy 'Lorem ipsum' text into the template

\usepackage{siunitx}

\usepackage{float}
\usepackage{wrapfig}
\usepackage{graphicx}

\usepackage{multirow}
\usepackage{booktabs}
\usepackage{array}

\usepackage{url}

\usepackage{hyperref}
\hypersetup{
    colorlinks,
    citecolor=black,
    filecolor=black,
    linkcolor=black,
    urlcolor=black
}
\usepackage{cleveref}
\usepackage{microtype}

\usepackage{todonotes}

\usepackage{titling}
\setlength{\droptitle}{-5em}

%----------------------------------------------------------------------------------------
% TITLE SECTION
%----------------------------------------------------------------------------------------

\newcommand{\horrule}[1]{\rule{\linewidth}{#1}} % Create horizontal rule command with 1 argument of height

\title{ 
\normalfont \normalsize 
% \textsc{university, school or department name} \\ [25pt] % Your university, school and/or department name(s)
\horrule{0.5pt} \\[0.4cm] % Thin top horizontal rule
\Huge Facial Detection and Localization: A Survey \\[0.5cm]
% \Large Computer Vision Project \\ [0.5cm] % The assignment title
\large COMP9517 - Semester 1, 2015 \\ [0.5cm]
\horrule{2pt} \\[0.5cm] % Thick bottom horizontal rule
}

\author{
  Louis Tiao \\
  (\texttt{3390558})
  \and
  Edward Lee\\
  (\texttt{3376371})
} % Your name

\date{\normalsize\today} % Today's date or a custom date

\begin{document}

\maketitle % Print the title


\section{Project Goals}

Facial detection and localization is an ubiquitous part of modern day computer vision with wide ranging applications from identification to entertainment. While a relatively easy task for humans and the human eye, facial detection and localization have been of particular interest towards computer vision researchers for many years.  As such, this project aims to outline and analyze a variety of different techniques for facial detection and localization in an experimental way. Doing so will allow a direct comparison between these algorithms can be made. Furthermore, analysis and experimentation will highlight real life conditions and challenging scenarios for facial detection algorithms and study the robustness with respect to these changing conditions and scenarios.

\section{Literature Survey}
Owing to the maturity of the field of facial detection, there is a large body of work and research which exists. We will be focusing on the following 4 techniques towards facial detection: Eigenfaces, Neural Networks, Support Vector Machines and Viola-Jones.

\subsection{Eigenfaces}
The eigenface technique originated from search for a low dimensional representation of a series of faces.
\textcolor{red}{incomplete}

\subsection{Neural Networks}
\textcolor{red}{incomplete}

\subsection{Support Vector Machines}
\textcolor{red}{incomplete}

\subsection{Viola Jones}
\textcolor{red}{incomplete}

\section{Project Decomposition} % subgoals




\section{Project Scope and specification} % what the examines and tests

\section{Algorithms and Design} 
%algorithms used, outline parsing databases - include justification

\section{Results}



\section{Conclusion}

\section{Group Subdivision}


\section{Quick Plan}

\textbf{Unconstrained Problem}: Given an image of a face, train to detect if there is a face, and give an answer as the position of the face in the scene.

\textbf{Constrained Problem}: Given an image of a face, train to detect if there is a face and give an answer as yes or no.

\subsection{Methods}
\begin{itemize}
  \item Eigenfaces - will only use with the constrained problem.
  \item SVMs - both problems
  \item Viola Jones - both problems
  \begin{itemize}
    \item original
    \item cascading
  \end{itemize}
  \item Neural networks
\end{itemize}

\subsection{Datasets}
\begin{enumerate}
  \item FDDB: http://vis-www.cs.umass.edu/fddb/fddb.pdf 

    - annotated ellipses with the Faces in the Wild dataset
    \textbf{Representative}

  \item Yale database (has different imaging conditions)
  \item Facial Expression Databse: http://www.kasrl.org/jaffe.html
  \item CMU: http://vasc.ri.cmu.edu/idb/html/face/

  Includes:
  \begin{itemize}
    \item Pose, Illumination, Expression (PIE) databse
    \item frontal images
    \item profile images
    \item facial expressions
  \end{itemize}
\end{enumerate}

\subsection{Factors:}
\begin{itemize}
  \item Location (in scene) - a face is positioned at different locations in an image
  \begin{itemize}
    \item \textbf{Dataset:} FDDB
    \item \textbf{Problems:} Both 
  \end{itemize}
  \item Pose - the face is tilted to some angle
  \begin{itemize}
    \item \textbf{Dataset:} FDDB
    \item \textbf{Problems:} Both 
  \end{itemize}
    \item Orientation
  \begin{itemize}
    \item \textbf{Dataset:} ??
    \item \textbf{Problems:} ??
  \end{itemize}
  \item Occlusion
  \begin{itemize}
    \item \textbf{Dataset:} Yale + Manual Occlusion via cropping face
    \item \textbf{Problems:} Constrained 
  \end{itemize}
  \item Imaging Condition
  \begin{itemize}
    \item \textbf{Dataset:} Yale
    \item \textbf{Problems:} Constrained
  \end{itemize}
  \item Structural Components

  \begin{itemize}
    \item \textbf{Dataset:} ??
    \item \textbf{Problems:} Constrained
  \end{itemize}
\end{itemize}

\subsection{Evaluation}

\begin{itemize}
  \item Unconstrained - FDDB metric - ratio of intersection of areas to union of areas of the ellipse and the detected face.
  \item Constrained - Percentage accuracy. Can include another dimension by considering the size of the training set, or parameters of the particular factor
\end{itemize}

\clearpage
\section{More Complete Plan}

\subsection{Training and Testing}
Training is done on the whole set of possible training examples for each algorithm, categorized into their particular factor below. Approximates 500 training examples of differing ratios to each factor exists.

Constrained testing involves testing the trained classifier's ability to detect whether or not a face exists in a given facial image with relatively uniform non-face backgrounds.. All testing images will be pictures of faces, tested with different factors. We are interested in the sensitivity or recall (TP / TP + FN) of each classifier.

The unconstrained testing evaluates each classifier in a real world setting with the Faces in the Wild Database and the FDDB annotations. Each classifier is given a score.

\textbf{Non-faces:} required for training all but eigenfaces.

  \begin{figure}[ht!]
    \centering
    \caption{Datasets for each stage}
    
    \begin{tabular}{r|l l l l}
      \textbf{Training} & Neural Net & Eigen-Faces & SVM & Viola-Jones \\
      \hline 
      Frontal default & Yale/FERET & Yale/FERET & Yale/FERET & FDDB \\
      Lighting & Yale & Yale & Yale & FDDB \\
      Facial Obstruction & FERET & FERET & FERET & FDDB \\
      % Expression & FERET & FERET & FERET & FDDB \\
      \hline \hline
      \textbf{Constrained Testing} & & & \\
      \hline
      Frontal default & Yale/FERET & Yale/FERET & Yale/FERET & Yale/FERET \\
      Occlusion & FERET + code & FERET + code & FERET + code & FERET + code \\
      Pose & Yale/FERET & Yale/FERET & Yale/FERET & Yale/FERET \\
      Lighting & Yale & Yale & Yale & Yale \\
      Facial Obstruction & FERET & FERET & FERET & FERET \\
      Expression & FERET/JAFFE & FERET/JAFFE & FERET/JAFFE & FERET/JAFFE \\
      \hline \hline
      \textbf{Unconstrained Testing} & & & \\
      \hline
      All factors & FDDB/CMU & FDDB/CMU & FDDB/CMU & FDDB/CMU \\
      
    \end{tabular}
    
  \end{figure}

\subsection{Evaluation - Constrained}
This is the first of two ways that an algorithm is evaluated. 


\subsection{Evaluation - Unconstrained}
Each of the trained algorithms in the previous instances will be evaluated among the Faces in the Wild database. All images will contain a face but the face is located in the scene in some manner.

%----------------------------------------------------------------------------------------

\end{document}