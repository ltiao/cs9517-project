%%%%%%%%%%%%%%%%%%%%%%%%%%%%%%%%%%%%%%%%%
% Short Sectioned Assignment
% LaTeX Template
% Version 1.0 (5/5/12)
%
% This template has been downloaded from:
% http://www.LaTeXTemplates.com
%
% Original author:
% Frits Wenneker (http://www.howtotex.com)
%
% License:
% CC BY-NC-SA 3.0 (http://creativecommons.org/licenses/by-nc-sa/3.0/)
%
%%%%%%%%%%%%%%%%%%%%%%%%%%%%%%%%%%%%%%%%%

%----------------------------------------------------------------------------------------
%	PACKAGES AND OTHER DOCUMENT CONFIGURATIONS
%----------------------------------------------------------------------------------------

\documentclass[11pt]{article} % A4 paper and 11pt font size

\usepackage[a4paper,margin=1in,footskip=.3in]{geometry}

\linespread{1.1} % Line spacing

\setlength\parindent{0pt} % Removes all indentation from paragraphs - comment this line for an assignment with lots of text
\setlength{\parskip}{6pt}

\usepackage{sectsty} % Allows customizing section commands
\allsectionsfont{\centering \normalfont\scshape} % Make all sections centered, the default font and small caps

\usepackage{fancyhdr} % Custom headers and footers
\pagestyle{fancyplain} % Makes all pages in the document conform to the custom headers and footers
\fancyhead{} % No page header - if you want one, create it in the same way as the footers below
\fancyfoot[L]{} % Empty left footer
\fancyfoot[C]{} % Empty center footer
\fancyfoot[R]{\thepage} % Page numbering for right footer
\renewcommand{\headrulewidth}{0pt} % Remove header underlines
\renewcommand{\footrulewidth}{0pt} % Remove footer underlines
\setlength{\headheight}{13.6pt} % Customize the height of the header

\usepackage[utf8]{inputenc}
\usepackage[T1]{fontenc} % Use 8-bit encoding that has 256 glyphs

\usepackage{fourier} % Use the Adobe Utopia font for the document - comment this line to return to the LaTeX default
\usepackage[english]{babel} % English language/hyphenation

\usepackage{amsmath,amsfonts,amsthm} % Math packages
\theoremstyle{plain}
\newtheorem{theorem}{Theorem}[section]
\newtheorem{corollary}{Corollary}[theorem]
\newtheorem{proposition}[theorem]{Proposition}
\newtheorem{lemma}[theorem]{Lemma}

\theoremstyle{definition}
\newtheorem{definition}{Definition}[section]

\theoremstyle{remark}
\newtheorem*{remark}{Remark}

\numberwithin{equation}{section} % Number equations within sections (i.e. 1.1, 1.2, 2.1, 2.2 instead of 1, 2, 3, 4)
\numberwithin{figure}{section} % Number figures within sections (i.e. 1.1, 1.2, 2.1, 2.2 instead of 1, 2, 3, 4)
\numberwithin{table}{section} % Number tables within sections (i.e. 1.1, 1.2, 2.1, 2.2 instead of 1, 2, 3, 4)

\usepackage{algpseudocode}
\usepackage{algorithm}

\usepackage{listings}

\usepackage{tikz}

\usepackage{natbib}
\bibliographystyle{plainnat}

\usepackage{lipsum} % Used for inserting dummy 'Lorem ipsum' text into the template

\usepackage{siunitx}

\usepackage{float}
\usepackage{wrapfig}
\usepackage{graphicx}

\usepackage{multirow}
\usepackage{booktabs}
\usepackage{array}

\usepackage{url}

\usepackage{hyperref}
\hypersetup{
    colorlinks,
    citecolor=black,
    filecolor=black,
    linkcolor=black,
    urlcolor=black
}
\usepackage{cleveref}
\usepackage{microtype}

%----------------------------------------------------------------------------------------
%	TITLE SECTION
%----------------------------------------------------------------------------------------

\newcommand{\horrule}[1]{\rule{\linewidth}{#1}} % Create horizontal rule command with 1 argument of height

\title{	
\normalfont \normalsize 
% \textsc{university, school or department name} \\ [25pt] % Your university, school and/or department name(s)
\horrule{0.5pt} \\[0.4cm] % Thin top horizontal rule
\Large Computer Vision Project (Part 1) \\ [0.1cm] % The assignment title
\large COMP9517 - Semester 1, 2015 \\ [0.2cm]
\horrule{2pt} \\[0.5cm] % Thick bottom horizontal rule
}

\author{
	Louis Tiao \\
	(\texttt{3390558})
	\and
	Edward Lee\\
	(\texttt{3xxxxxx})
} % Your name

\date{\normalsize\today} % Today's date or a custom date

\begin{document}

\maketitle % Print the title

%----------------------------------------------------------------------------------------
%	PROBLEM 1
%----------------------------------------------------------------------------------------

\section{Introduction}

In Part 1 of the project, we are tasked with implementing software to track a number 
of (planar) objects in sequential video frames. In other words, given a number of 
reference images containing planar objects, we are required to estimate the motion 
models that describe how the objects are changing in the video sequence over time, 
in terms of translation, rotation, scale and other homographic transformations. We 
present a high-level overview of our approach in \cref{sec:approach} and discuss the 
results in \cref{sec:results}.

\section{Approach} \label{sec:approach}

We use the conventional feature-based methods to approach this task. Namely, we find
the correspondences between distinctive features in the video frames and the reference 
images, such as corners, edges, blobs, etc. and thereby fit a homographic transformation 
to the corresponding points in the video frame and reference image. This process is
broken down into the following subtasks, and we describe our approach for each one.

\subsection{Detection and Description}

To detect distinctive features in images and compute their description vectors 
(descriptors), we first experimented with the class of algorithms based on scale-space 
extrema detection: \emph{Scale-Invariant Feature Transform} (SIFT) and \emph{Speeded-Up 
Robust Features} (SURF), the latter being an improvement and optimization of the former. 

These algorithms encompass the tasks of detecting features and computing descriptors
in a way that is both scale and rotation invariant. In practice, we found SIFT offered
very good performance in terms of accuracy, but was impractically slow for real-time 
application. Although SURF offered better performance in terms of speed, it did so at
the cost of some accuracy and was ultimately still too slow for real-time application.

Next, we experimented with (a combination of) other feature detectors and descriptor 
extractors, namely: \emph{Features from Accelerated Segment Test} (FAST), which is
solely a feature detector, \emph{Binary Robust Independent Elementary Features} (BRIEF),
which is solely a descriptor extractor, and finally, \emph{Oriented FAST and Rotated BRIEF}
(ORB), which (as the name suggests) is an amalgamation of FAST and BRIEF with 
modifications to support rotation invariance.


In \cref{sec:results}

\subsection{Matching}

\begin{itemize}
	\item Brute Force (\texttt{NORM\_HAMMING})
	\item Approximate Nearest Neighbors (kd-Tree, Locality Sensitive Hashing) 
\end{itemize}

Outlier rejection

\subsection{Transformation}

\texttt{findHomography} (RANSAC)
\texttt{perspectiveTransform}

\section{Results} \label{sec:results}



%----------------------------------------------------------------------------------------

\end{document}